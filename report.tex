\documentclass[a4paper, 11pt]{article}
\usepackage{comment} % enables the use of multi-line comments (\ifx \fi)
\usepackage{lipsum} %This package just generates Lorem Ipsum filler text.
\usepackage{fullpage} % changes the margin

\begin{document}
%Header-Make sure you update this information!!!!
\noindent
\LARGE\textbf{Project Report} \hfill \textbf{RU-Exploring} \\ \\
\normalsize Computer Networks (CS-552) \hfill Saurabh Singh (ss2716)\\
Prof. Badri Nath \hfill Manjusha Ray (myr14)\\
\normalsize{TA: Eduard Renard} \hfill Alok Singh (as2509)

\section*{Abstract}
RU-Exploring is a travel app which allows users to browse through various places of interest across Rutgers University-New Brunswick.\\
The application fully implements the powerful and at the moment very hot concept of Deeplinking on mobile platform. Mobile deep links let users share content that's within an app.%Citation\cite{Flueck}

\section*{Deeplinking}
RU-Exploring deep-links to other native apps for seamless user experience while exploring places of interest. The app implements deep-linking at three levels:
\begin{itemize}
\item
\textbf{To the native UBER application:}\\RU-Exploring allows user to book a ride with UBER directly on a click of a button. The user is presented with an option of booking a ride with UBER on every content page. The information about the place he wants to book the ride to is preserved in the deep link and is passed to the native UBER application before it is invoked. This allows the UBER application to do some prepossessing and automatically set the destination as the place from the RU-Exploring application. This enhances the user experience many folds. In case the user does not have the native UBER application installed, he is taken to the Application installation page.
\item
\textbf{To the native Google maps application:}\\The user is presented with an option of locating the place on google maps. The information about the place is preserved in the deeplink and is passed to the native google maps application before it is invoked. This allows the google maps application to do some prepossessing and automatically set the destination as the place from the RU-Exploring application and show the optimal routes between user’s current location to the place. This is a much better user experience compared to the traditional approach of taking the user to the maps on web view in the same application. In case the user does not have the native maps application installed, he is taken to the Application installation page.
\item
\textbf{To RU-Exploring using Branch.io:}\\This is the most powerful form of Deeplinking the application implements. We allow users to share specific content (places) in the application with the users of the application. The share option allows user to pick any of the popular messaging or emailing applications to share the content. On the receiver’s device, the click to the link invokes the RUExploring application and opens up the specific content that was shared if it is already installed, or redirects the user to the application installation page. The significant feature of the application I that the information in the deeplinks is preserved through the installation process and the user is taken to the specific content that was shared.  The feature is called content sharing and this is one of the most effective ways of application promotion.
\end{itemize}

\ifx
\begin{itemize}
		\item item1k
		\item item2
	\end{itemize}
\fi


% \section*{Attachments}
% %Make sure to change these
% Lab Notes, HelloWorld.ic, FooBar.ic
% %\fi %comment me out

% \begin{thebibliography}{9}
% \bibitem{Robotics} Fred G. Martin \emph{Robotics Explorations: A Hands-On Introduction to Engineering}. New Jersey: Prentice Hall.
% \bibitem{Flueck}  Flueck, Alexander J. 2005. \emph{ECE 100}[online]. Chicago: Illinois Institute of Technology, Electrical and Computer Engineering Department, 2005 [cited 30
% August 2005]. Available from World Wide Web: (http://www.ece.iit.edu/~flueck/ece100).
% \end{thebibliography}

\end{document}
